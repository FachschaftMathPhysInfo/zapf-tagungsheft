\documentclass[a5paper]{scrbook}
% !TEX ROOT = main.tex

\usepackage[ngerman]{babel}
\usepackage[utf8]{inputenc}
\usepackage[T1]{fontenc}
\usepackage{microtype}
\usepackage{amsmath}
\usepackage{eurosym}
\usepackage[pdftex]{graphicx}
\usepackage{fancyhdr}
\usepackage{blindtext}
\usepackage{geometry}
\usepackage{url}

% Suche nach Grafiken in ./media und .:
\graphicspath{{./media/}{./}}

% Satzspiegel
\geometry{inner=20mm, outer=15mm, top=15mm, bottom=25mm, heightrounded, marginparwidth=37mm, marginparsep=5mm}

%Headlines
\pagestyle{fancy}
\fancyhf{}
\fancyhead[LE]{\leftmark}
\fancyhead[RO]{\rightmark}
\fancyfoot[RO,LE]{\thepage}


%zusätzliche Packages nur für den Mantelbogen
\usepackage[absolute]{textpos}  % Zum Positionieren der Grafiken
\usepackage{rotating}           % Zum Drehen von Text
\setlength{\parindent}{0pt}

\begin{document}

% VORDERSEITE, AUSSEN %
\pagestyle{empty}
\vspace*{30mm} \centering \fontsize{40}{48} \textbf{Tagungsheft}
\normalsize

\begin{textblock*}{148mm}[0,0](0mm,100mm)
    \includegraphics[width=148mm]{logo_breit} 
\end{textblock*}
  
% TeX denkt, dass die Seite noch leer ist und macht deswegen keinen
% pagebreak. Das \null täuscht TeX und alle sind glücklich (bis auf die
% Perfektionisten natürlich)
\null
\newpage

% VORDERSEITE, INNEN %
\vspace*{\fill}
    \begin{tabular*}{0.77\textwidth}{ll}
        \multicolumn{2}{l}{
            \parbox{0.77\textwidth}{
                Der Redaktionsschluss für diesen Text war am ENTER TEMRIN HIER. Wir freuen uns
                sehr über Kommentare, Anregungen, Verbesserungsvorschläge,
                Mitarbeit und Kuchen -- melde dich bei
                \href{mailto:akzapf@mathphys.stura.uni-heidelberg.de}{akzapf@mathphys.stura.uni-heidelberg.de}.
            }
            \vspace{5cm}
        }\\
        \textbf{Impressum} &\\
        Herausgeber & Studienfachschaft Physik, Universität Heidelberg \\
        & Im Neuenheimer Feld 205, Raum 01.301\\
        & 69120 Heidelberg\\
        V.\,i.\,S.\,d.\,P. & Jan Gräfje\\
        & Freiburger Straße 1A\\
        & 69126 Heidelberg\\
    \end{tabular*}

    \vfill

    \begin{textblock*}{203mm}[0,1](-7mm,290mm)
        \begin{flushright}
            \footnotesize
            \href{http://mathphys.info}{Coverbild: URHEBER} \href{http://creativecommons.org/licenses/by-sa/4.0/}{(CC-BY-SA)}\\  
        \end{flushright}
    \end{textblock*}



\null
\newpage

% RÜCKSEITE, INNEN %
\pagestyle{empty}

\null
\newpage

% RÜCKSEITE, AUSSEN %
\pagestyle{empty}


\end{document}
