% !TEX TS-program = pdflatex
% !TEX encoding = UTF-8 Unicode
% !TEX ROOT = main.tex

\section{Anlaufstellen}

\subsection{Anmeldung}
\faMapPin \quad Foyer der Jugendherberge Heidelberg\\
\faClockO \quad \\ \todo{Öffnungszeiten der Anmeldung} % bis Plenumsbeginn 18:00, danach Tagungsbüro

\noindent Bei der Anmeldung bekommt man seine Tagungstasche und einen Badge. % Außerdem muss hier noch der Teilnehmerbeitrag bezahlt werden, falls noch nicht geschehen.

\subsection{Tagungsbüro}
\faPhone \quad +49 6221 54 ???\\ \todo{Nummer vom PI}
\faMapPin \quad INF 226 (Physikalisches Institut), Raum 00.210\\ % Tagungsbüro ist wo anders, nicht =Orga-Büro
\faClockO \quad \\ \todo{Öffnungszeiten des Tagungsbüros einfügen} % 6:00 - 2:00

\noindent Das Tagungsbüro sollte immer die erste Anlaufstelle sein. Dort wird euch kompetent weitergeholfen, egal was ihr wollt oder braucht.

\subsection{Orga-Büro}
\faPhone \quad +49 6221 54 ???\\ \todo{Nummer vom PI}
\faMapPin \quad INF 226 (Physikalisches Institut), Raum 00.210\\
\faClockO \quad \\ \todo{Nicht durchgängig besetzt} % doch

\noindent Bei wichtigen Dingen kann man im Orga Büro vorbei kommen. % Grade nachts, wenn das Tagungsbüro nicht besetzt ist.

\subsection{Vertrauenspersonen}
\faPhone \quad +49 ???? ????\\ \todo{Nummer von Vertrauenspersonen} % Irina? Juliana?
\faUsers \quad \\ \todo{Namen der Vertrauenspersonen}

\noindent \todo[inline]{Text zu Vertrauenspersonen}
