% !TEX TS-program = pdflatex
% !TEX encoding = UTF-8 Unicode
% !TEX ROOT = main.tex

\section{Schlafen \& Duschen}
Ihr habt das große Glück, dass die Heidelberger Stadtverwaltung ihre öffentlichen Räumlichkeiten nicht zum Zwecke von Übernachtungen an uns vermieten will. Das heißt konkret, dass ihr in der örtlichen Jugendherbege und in Gemeinschaftsräumen von Wohnheimen des Studierendenwerks untergebracht werdet. Für euren Aufenthalt ergeben sich somit eine Vereinfachungen im Gegensatz zur Unterbringung in Turnhallen.\\
Besonders für diejenigen unter euch, die in der Jugendherberge übernachten dürfen.
\begin{itemize}
  \item Es gibt richtige Matratzen für euren Schönheitsschlaf
  \item Eure Schlafsachen und das Gepäck könnt ihr bis Sonntagmorgen auf den Zimmern lassen
  \item Die Zimmer können abgeschlossen werden
  \item Nachts gibt es genug Steckdosen
  \item Duschen für alle!
\end{itemize}
Ihr schwer arbeitenden ZaPFika habt euch diesen Luxus hart erarbeitet und wir sind uns mit unseren Sponsoren einig, dass ihr es auch verdient! \\
%\todo[inline]{Eventuell noch seriöse Details zur Unterbrinung (Räumen der Zimmer, Zimmeraufteilung etc.) eintragen}

\subsection{Unterbringung in den Gemeinschaftsräumen}
Vierzig von euch haben die Ehre in den sogenannten Notunterkünften des Studierendenwerks unterzukommen, die uns dankenswerterweise zur Verfügung gestellt werden. 
Ihr sollt jedoch keine Not erleiden, ganz im Gegenteil, in einem Raum dürfen sich jeweils zehn Personen das Glück teilen. Es handelt sich um die Gebäude INF 693, 694, 695 \& 696.\\Zu den Schlafenszeiten werden dort Helfika bereit stehen um euch zu empfangen und euch die Türen öffnen, bis in die frühen Morgenstunden hinein.\\
Eine Duschmöglichkeit direkt vor Ort besteht nicht, jedoch werden euch in einem KIP-Nebengebäude zu ausgewählten Zeiten Duschen zur Verfügung stehen.\\
Bitte beachtet die Hausordnung in den Räumlichkeiten des Studierendenwerks, das bedeutet vor allem die \textbf{Nachtruhe von 22:00 bis 07:00} zu respektieren und euch dementsprechend zu diesen Uhrzeiten besonders ruhig zu verhalten.\\
Sollten während der Tagung Probleme und Fragen auftreten meldet euch tagsüber beim  \textbf{Tagungsbüro} oder abends direkt beim Außenposten in einem der Wohnheime.

\subsection{Sanitäranlagen während der Tagung}

Zum Vergolden der Toiletten und Armaturen hatten wir leider nicht genug Zeit, aber es gibt auf jeden Fall genug Stille Orte in Reichweite der Tagungsräume und auf den Zimmern. Auf dem Gebäuderaumplan sind mehrere Alternativen angegeben. Wenn du dann mal bemerken solltest, dass die Dinge des täglichen Bedarfs zur Neige gehen, melde dich am besten kurz im Tagungsbüro. Dann können wir entsprechend reagieren. Das Nachfüllen passiert leider noch nicht automatisiert bei uns - das Robotiklabor ist erst in der Alpha-Phase.
