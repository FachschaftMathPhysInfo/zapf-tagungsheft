% !TEX TS-program = pdflatex
% !TEX encoding = UTF-8 Unicode
% !TEX ROOT = main.tex

\newcommand*{\fett}[1]{\textbf{#1}}

\chapter{Glossar}

\begin{description}
	\item[AK] In \fett{A}rbeits\fett{k}reisen werden alle Themen der ZaPF bearbeitet, entweder in Form von offenen Diskussionen, inhaltlichem Austasuch, dem Abarbeiten konkreter Arbeitsaufträge oder dem Verfassen eines Antrags für das Plenum. Alle AKs werden protokolliert.
	\item[Akkreditierung] Die ZaPF kann Vertreter*innen in den studentischen Akkreditierungspool entsenden, die dann Studiengänge akkreditieren können. Daher beschäftigt sich die ZaPF fortlaufen mit dem Akkreditierungswesen in Deutschland.\footnote{Begriffserklärung: siehe zapf.wiki/Kategorie:Akkreditierung}
	\item[Alter Sack] 
	\item[Austausch-AK] description
	\item[Bielefeld] Gibt es nicht
	\item[Bier-AK] description
	\item[CHE] description
	\item[Ente] description
	\item[Ente] description
	\item[ewiges Frühstück] description
	\item[Folge-AK] description
	\item[GO] description
	\item[GO-Antrag]
	\item[INF] \fett{I}m \fett{N}euenheimer \fett{F}eld ist der Straßenname aller Gebäude auf diesem Campus. Oft steht die Gebäudenummer groß auf dem Gebäude.
	\item[jDPG] description
	\item[KIP] \fett{K}irchhoff-\fett{I}nstitut für \fett{P}hysik (INF 226), Gebäude in dem die Plena und die meisten AKs stattfinden 
	\item[KIP2] description
	\item[KMK] Kultusministerkonferenz
	\item[KommGrem] description
	\item[Kuschel-AK] description
	\item[LEuTe] description
	\item[LRK] Landesrektorenkonferenz
	\item[MeTaFa] description
	\item[Plenum] description
	\item[Positionspapier] description
	\item[Postersession] description
	\item[Resolution] 
	\item[Reader] description
	\item[Satzung] description
	\item[StAPF] description
	\item[TO] description
	\item[TOP] description
	\item[TOPF] description
	\item[Vertrauensperon] description
	\item[ZaPF] description
	\item[ZaPF e.V.] description
	\item[ZaPF-Wiki] description
	\item[ZäPFchen] description
	\item[ZaPFika] description
\end{description}