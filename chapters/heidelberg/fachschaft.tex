% !TEX TS-program = pdflatex
% !TEX encoding = UTF-8 Unicode
% !TEX ROOT = ../../main.tex

\section{Fachschaft MathPhysInfo}

Die Uni Heidelberg hat die größte Physikfakultät Deutschlands, trotzdem gelingt es uns Fachschaftlern immer wieder neue Veranstaltungen für alle Studierenden zu planen. Um den Organisations- wie den Verwaltungsaufwand trotzdem möglichst gering zu halten, haben sich die Studienschaften Mathematik, Informatik und Physik zu der Fachschaft MathPhysInfo zusammengeschlossen. Fachlich mögen wir getrennt sein, aber alle anderen Angelegenheiten wie Räumlichkeiten und Freizeitveranstaltungen veranstalten wir gemeinsam. Dazu zählt auch die diesjährige ZaPF, wundert euch also nicht, wenn ihr unter den Helfenden den ein oder anderen Mathematika und Informatika findet.

Wir sitzen, wie Fachschaften das so tun, in einigen Gremien, veranstalten den Vorkurs der MathInffakultät und helfen bei dem der Physikfakultät, für letztere organisieren wir auch die Evaluation der Lehrveranstaltungen. Außerdem haben wir wöchentlich mehr oder weniger sinnvolle Sitzungen und irgendwer ist eigentlich immer im Fachschaftsraum, weswegen dieser auch schon als 20--Personen--WG bezeichnet wurde.

Wir können aber nicht nur arbeiten, sondern gleichzeitig auch feiern, denn wir organisieren die größte Studiparty Heidelbergs, zusammen mit der Fachschaft Theologie. Um gleich mal mit den Gerüchten aufzuräumen, diese Kombination ist \textit{nicht} durch den hohen Frauenanteil der Theologen entstanden.
%\begin{wrapfigure}{r}{0.4\textwidth}
%\includegraphics[width=\linewidth]{media/mathphysinfologo}
%\vspace*{-20pt}
%\end{wrapfigure}
Wer es etwas gemütlicher mag kommt zu der monatlichen Cafétenfete oder zu den Spieleabenden, und wer dann noch mehr Zeit mit der Fachschaft verbringen möchte, begleitet uns auf das Fachschaftswochenende in den Odenwald, um dort zu diskutieren und einander besser kennenzulernen.

Falls ihr noch mehr zu uns und was wir so machen, wissen wollt, findet ihr das auf unserer Website \url{www.mathphys.info} und Social Media: 
\begin{itemize}
\item[\faFacebookSquare] Facebook: \href{https://www.facebook.com/fachschaft.mathphysinfo}{@fachschaft.mathphysinfo}
\item[\faTwitterSquare] Twitter: \href{https://twitter.com/MathPhysInfo}{@MathPhysInfo}
\item[\faInstagram] Instagram: \href{https://www.instagram.com/mathphysinfo/}{mathphysinfo}
\end{itemize}
\begin{figure}[h]
\centering
\includegraphics[width=0.5\linewidth]{media/mathphysinfologo}
\end{figure}