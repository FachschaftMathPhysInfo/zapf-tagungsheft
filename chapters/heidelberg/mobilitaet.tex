\section{Mobilität}
Durch die drei Campi in Heidelberg haben nicht alle Studierenden den Vorteil mitten in der Altstadt studieren zu können. Gerade die Naturwissenschaften mit ihren großen Laboren brauchten neuen Platz und mussten notgedrungen "auswandern''. Unser Campus ist das Neuenheimer Feld und in sich auch sehr schön gestaltet, ganz nach den \textit{Kunst im Bau} Grundsätze. Besonders abends und als Besuchikon will man doch gerne mal in die historische Altstadt, dem originalen Heidelberg.

Für Einheimische ist das Verkehrsmittel der Wahl natürlich das Fahrrad, aber das wird wohl kaum einer von euch extra mitgenommen haben. Daher haben wir euch ein paar Alternativen  rausgesucht.
  %Auf den letzten ZaPFen haben wir festgestellt, dass die Mobilitätstickets von den meisten nicht sehr häufig genutzt wurden. Darum haben wir in Heidelberg entschieden euch dieses Mal kein teures Ticket für das gesamte lange Wochenende zu organisieren.
  % Wir fangen gar nicht erst an, uns für Orga-Entscheidungen zu rechtfertigen. Es gibt kein verbrieftes ZaPF-Recht auf ein tolles ÖPNV-Ticket!
Für die Exkursionen werden euch die entsprechenenden Nahverkehrstickets selbstverständlich kostenfrei zur Verfügung gestellt.

\subsection{öffentlicher Nahverkehr}
Wer neben der Kneipentour hin und wieder gerne mal in die Altstadt möchte, der hatte bei     der Anmeldung schon die Möglichkeit, sich eine HeidelbergCard zu kaufen. Es ist unsere günstige Empfehlung, um vor Ort stressfrei von A nach B zu kommen. Und ihr dürft sogar kostenfrei die Bergbahn auf den Königstuhl benutzen. Falls ihr euch doch spontan überlegt, abends mal die Bahnen zu nehmen, kann direkt in der Jugendherberge ein Ticket für 19 \euro für die vier Tage erworben werden oder ihr holt euch eben je nach Bedarf Fahrscheine an den Automaten an den Bahnhaltestellen oder bei der busführenden Person. Das ist auch ganz praktisch, wenn ihr später am Abend mal zu faul seid, doch noch den ganzen Weg von den Kneipen zur Herberge zu laufen. Wahrscheinlich ist es für euch alle am schlausten, die HeidelbergCard erst Donnerstag zu kaufen, damit ihr Sonntag nach dem Endplenum noch die Altstadt und den Königstuhl besuchen könnt, wenn ihr mit all den Plenen und AKs noch nicht genug zu tun habt. Wenn ihr denn überhaupt Busse und Bahnen nötig habt, zu Fuß ist man auch recht zügig unterwegs.

Für die Lauffaulen gibt es sogar zwei verschiedene Busse, die von der Jugendherberge bis in die Altstadt fahren. Die Linie \textbf{31} fährt von der Haltestelle \texttt{Jugendherberge} über das \texttt{Bunsengymnasium} bis zum \texttt{Universitätsplatz} und wieder zurück mit der Zielhaltestelle \texttt{Chirurgische Klinik}. Das Bunsengymnasium ist von den Arbeitsräumen die nächste Haltestelle. Stattdessen könnt ihr auch die Linie \textbf{32} nehmen, je nachdem welche grade passender ist. Sie fährt von der \texttt{Jugendherberge} auch zum \texttt{Universitätsplatz} und zurück bis zur \texttt{Kopfklinik}. Jedoch könnt ihr dann nicht am \texttt{Bunsengymnasium} ein- oder aussteigen.
       % \todo{Busnetzplan Linien 31, 32}

\subsection{VRNnextbike}

\begin{figure}[t]
  \includegraphics[width=1.0\textwidth]{chapters/heidelberg/nextbike}
  \caption*{Verleihstellen VRNnextbike}
  \label{nextbike}
\end{figure}

Der regionale Verkehrsverbund bietet auch Mietfahrräder an. Das ist eine super Gelegenheit  für kleines Geld schnell zur Altstadt, zum Bahnhof oder wozu auch immer zu kommen.  Praktisch ist es besonders dann, wenn ihr am Start- und Zielort eine Verleihstation habt. Leider können wir euch hierfür kein Zeitticket anbieten, es gibt schlichtweg keins. Die Registrierung erfolgt allerdings kostenfrei\footnote{\url{http://www.vrnnextbike.de/de/information/}}. Für 1 \euro \, pro halber Stunde (maximal 9 \euro \, pro Ausleihe) gibt es ein verkehrstüchtiges Fahrrad und das sportliche Workout inklusive.
    % Gerade wenn ihr in der etwas abseits gelegenen Jugendherberge untergebracht seit,
    % macht das durchaus Sinn, da dort auch einige Räder direkt vor der Tür auf euch warten.
      %  \todo{VRNextbike direkt an der JH?}



%Und nicht zuletzt ist Heidelberg eine sehr radfahrerfreundliche Stadt.