\section{Mobilität}
  \subsection{öffentlicher Nahverkehr}
    Auf den letzten ZaPFen haben wir festgestellt, dass die Mobilitätstickets von den meisten
    nicht sehr häufig genutzt wurden. Darum haben wir in Heidelberg entschieden euch
    dieses Mal kein Ticket für das lange Wochenende zu organisieren.
    Wer trotzdem hin und wieder gerne mal in die Altstadt möchte, der hatte bei
    der Anmeldung schon die Möglichkeit, sich eine HeidelbergCard zu kaufen.
    Es ist unsere Empfehlung, um in vor Ort stresfrei von A nach B zu kommen.
    Und ihr dürft sogar kostenfrei die Bergbahn auf den Königstuhl benutzen.
    Falls ihr euch doch spontan überlegt, abends mal die Bahnen zu nehmen, kann
    direkt in der Jugendherberge (Tiergartenstraße 5) ein Ticket 19\euro für die vier Tage
    erworben werden oder ihr holt euch eben je nach Bedarf Fahrscheine an den Automaten
    an den Bahnhaltestellen oder beim Fahrer.
    Wahrscheinlich ist es für euch alle am schlausten, die HeidelbergCard erst Donnerstag zu kaufen,
    damit ihr Sonntag nach dem Endplenum noch die Altstadt und den Königstuhl besuchen könnt.
  \subsection{VRNnextbike}
    Der regionale Verkehrsverbund bietet auch Mietfahrräder an. Das ist eine super Gelegenheit
    für kleines Geld schnell in die Altstadt, den Bahnhof oder was auch immer zu kommen.
    Praktisch ist es besonders dann, wenn ihr am Start- und Zielort eine Verleihstation habt.
    Leider können wir euch hierfür kein Zeitticket anbieten, es gibt schlichtweg keins.
    Die Registrierung erfolgt allerdings kostenfrei und ist auf der folgenden Seite genauer beschrieben
    (http://www.vrnnextbike.de/de/information/).
    Für 1\euro \, pro halber Stunde (maximal 9\euro \, pro Ausleihe) gibt es ein verkehrstüchtiges
    Fahrrad und das sportliche Workout ist inklusive.
